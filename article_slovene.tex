\documentclass{styles/stucosrec_clanek}

% for testing pouposes
\usepackage{lipsum}

\title{Example to show how classes work}
\author{
	David S.~Hippocampus\thanks{Use footnote for providing further information about author (webpage, alternative address)---\emph{not} for acknowledging funding agencies.} \\
	Department of Computer Science\\
	Cranberry-Lemon University\\
	Pittsburgh, PA 15213 \\
	\texttt{hippo@cs.cranberry-lemon.edu} \\
	\And
	Elias D.~Striatum \thanks{He was working too} \\
	Department of Electrical Engineering\\
	Mount-Sheikh University\\
	Santa Narimana, Levand \\
	\texttt{stariate@ee.mount-sheikh.edu} \\
	\And
	Elias D.~Striatum \thanks{He was working too} \\
	Department of Electrical Engineering\\
	Mount-Sheikh University\\
	Santa Narimana, Levand \\
	\texttt{stariate@ee.mount-sheikh.edu} \\
	%% \And
	%% Coauthor \\
	%% Affiliation \\
	%% Address \\
	%% \texttt{email} \\
	%% \And
	%% Coauthor \\
	%% Affiliation \\
	%% Address \\
	%% \texttt{email} \\
}
\date{February 2021}

\begin{document}
	
	\begin{abstract}
		\lipsum[1]
	\end{abstract}
		
	% keywords can be removed
	\keywords{First keyword \and Second keyword \and More}
		
	\section{Introduction}
	\lipsum[2]
	\lipsum[3]
	
	
	\section{Headings: first level}
	\label{sec:headings}
	
	\lipsum[4] See Section \ref{sec:headings}.
	
	\subsection{Headings: second level}
	\lipsum[5]
	\begin{equation}
		\xi _{ij}(t)=P(x_{t}=i,x_{t+1}=j|y,v,w;\theta)= {\frac {\alpha _{i}(t)a^{w_t}_{ij}\beta _{j}(t+1)b^{v_{t+1}}_{j}(y_{t+1})}{\sum _{i=1}^{N} \sum _{j=1}^{N} \alpha _{i}(t)a^{w_t}_{ij}\beta _{j}(t+1)b^{v_{t+1}}_{j}(y_{t+1})}}
	\end{equation}
	
	\subsubsection{Headings: third level}
	\lipsum[6]
	
	\paragraph{Paragraph}
	\lipsum[7]	
	
	\section{Examples of citations, figures, tables, references}
	\label{sec:others}
	
	\subsection{Citations}
	Citations use \verb+natbib+. The documentation may be found at
	\begin{center}
		\url{http://mirrors.ctan.org/macros/latex/contrib/natbib/natnotes.pdf}
	\end{center}
	
	Here is an example usage of the two main commands (\verb+citet+ and \verb+citep+): Some people thought a thing \cite{kour2014real, hadash2018estimate} but other people thought something else \cite{kour2014fast}. Many people have speculated that if we knew exactly why \cite{kour2014fast} thought this $\dots$
	
	\subsection{Figures}
	\lipsum[10]
	See Figure~\ref{fig:fig1}. Here is how you add footnotes. \footnote{Sample of the first footnote.}
	\lipsum[11]
	
	\begin{figure}[H]
		\centering
		\fbox{\rule[-.5cm]{4cm}{4cm} \rule[-.5cm]{4cm}{0cm}}
		\caption{Sample figure caption.}
		\label{fig:fig1}
	\end{figure}

	\subsection{Tables}
	See awesome Table~\ref{tab:table}.
	
	The documentation for \verb+booktabs+ (`Publication quality tables in LaTeX') is available from:
	\begin{center}
		\url{https://www.ctan.org/pkg/booktabs}
	\end{center}	
	
	\begin{table*}
		\centering
		\caption{Some Typical Commands}
		\label{tab:table}
		\begin{tabular}{|c|c|l|} \hline
			Command&A Number&Comments\\ \hline
			\texttt{{\char'134}alignauthor} & 100& Author alignment\\ \hline
			\texttt{{\char'134}numberofauthors}& 200& Author enumeration\\ \hline
			\texttt{{\char'134}table}& 300 & For tables\\ \hline
			\texttt{{\char'134}table*}& 400& For wider tables\\ \hline
		\end{tabular}
	\end{table*}

	\begin{table}[H]
		\centering
		\caption{Frequency of Special Characters}
		\begin{tabular}{|c|c|l|} \hline
			Non-English or Math&Frequency&Comments\\ \hline
			\O & 1 in 1,000& For Swedish names\\ \hline
			$\pi$ & 1 in 5& Common in math\\ \hline
			\$ & 4 in 5 & Used in business\\ \hline
			$\Psi^2_1$ & 1 in 40,000& Unexplained usage\\
			\hline
		\end{tabular}
	\end{table}
	
	\subsection{Lists}
	\begin{itemize}
		\item Lorem ipsum dolor sit amet
		\item consectetur adipiscing elit.
		\item Aliquam dignissim blandit est, in dictum tortor gravida eget. In ac rutrum magna.
	\end{itemize}
	
	\bibliography{references}

\end{document}
